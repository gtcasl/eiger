
\chapter{Instrumentation Reuse}\label{sec:reuse}
In benchmarking, code is sometimes manually (or mechanically) translated
between C++ and Fortran for various reasons. The macro design of lwperf
is intended to support cut-and-paste reuse of site markup
across language boundaries. The Fortran instrumentation tolerates but
does not require trailing semicolons on macro uses, so PERFLOG/PERFSTOP uses
from C++ can be pasted into equivalent Fortran sites. The only modifications
needed being to insert or remove the argument count suffix to obtain the equivalent macro name.

Similarly, LWPERF macro calls from instrumented applications usually may be pasted directly into the equivalent skeleton applications written using the SSTmacro performance modeling toolkit\cite{ref:sstmac}. When the mapping is one-to-one, the performance data gathered from the application and the performance model may be compared element by element.

