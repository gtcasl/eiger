
\chapter{Implementation}\label{sec:impl}
 The collector is aware that node-locally there may be a number of OMP threads or MPI ranks relevant to performance modeling and expects the user to pass in that information as part of the initialization or input parameter data at measurement sites. In the case of multiple MPI ranks on the node, each log file name is suffixed by the MPI communicator rank and size passed in with an init call. No interprocess communications (except those of connecting to the Eiger database or the output filesystem) are introduced by using the collector.

The logger backends (eigerformatter, csvformatter) support eiger, fakeeiger, or CSV. In the eiger cases, the default is to generate CSV files also. This default can be changed by passing -D\_EIGER\_NOCSV to the c++ compiler.

The backends can also generate text to the screen (in csv format) if the screen option is turned on by passing -D\_LWPERF\_SCREEN to the c++ compiler.


\section{Dependencies}
\begin{itemize}
\item[Eiger] The collectors depend on libmpieiger or libmpifakeeiger if -D\_USE\_EIGER is applied. Eiger is not needed if only CSV collection is used.
\item[MPI] From MPI, the collectors use only MPI\_Wtime. This may be easily replaced with boost or any other available and acceptable high resolution clock difference. The MPI compiler wrappers are used because of this dependency.
\item[C++] STL classes and variadic macro support.
\item[Fortran] Basic macro substitution and the bind(c) feature, which is available in gfortran and most other fortrans as of 2011. Formally, bind(c) is part of the Fortran 2003 specification. Variadic macro support is not required for the Fortran binding.
\item[C] Variadic macro support
\end{itemize}



