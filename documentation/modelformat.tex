\section{Model File Format}
\label{sec:modelfile}
This section describes the text format for model files generated by Eiger. The function of these files are to easily marshall and unmarshall performance model data for use in other simulators, such as SST/macro. The goal is ease of use and readability over compactness and bit-correctness.
\subsection{File Format}
The model captures each effect required to compute the value of a prediction. It begins with the combined principal component analysis transformations from the application and machine metrics, followed by the loadings of each predictor function, followed by an encoding of the functions themselves. The last is a list of each input metric the model requires, printed in order, i.e. the first metric listed is considered the first dimension for PCA, the second metric is the second dimension, and so on. The file is formatted as follows. Each element is separated by a newline character.
	\begin{itemize}
	\item \texttt{[}\# rows in PCA matrix,\# cols in PCA matrix\texttt{]}$((val_{00},val_{01},...),(val_{10},val_{11}...),...)$
	\item \texttt{[}\# functions in model\texttt{]}($\beta_0,\beta_1,...$)
	\item encoded functions in model, separated by newline
	\item name of each input metric for the model, ordered, separated by newline
	\end{itemize}

\subsection{Function Encodings}
Each predictor function must be reevaluated when a prediction is being made. To do so, an encoding is established representing the function evaluated. This is inherently tied to the type of regression being performed, as well as the model pool used to feed the regression process. For parametric linear regressions, the function encoding is as follows.
	\begin{table}[h]
	\centering
	\begin{tabular}{|l|l|}
	\hline
	Encoding & Function \\
	\hline
	\texttt{0} & $f(\mathbf{x},i) = 1$ \\
	\hline
	\texttt{1 i n} & $f(\mathbf{x},i,n) = x_i^n$ \\
	\hline
	\texttt{2 i j} & $f(\mathbf{x},i,j) = x_i * x_j$ \\
	\hline
	\texttt{3 i} & $f(\mathbf{x},i) = \sqrt{x_i}$ \\
	\hline
	\texttt{4 i} & $f(\mathbf{x},i) = \log_2(x_i)$ \\
	\hline
	\texttt{5 i} & $f(\mathbf{x},i) = 1/x_i$ \\
	\hline
	\end{tabular}
	\end{table}
