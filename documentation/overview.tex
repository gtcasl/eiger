\section{Overview}
\label{sec:over}
\subsection{Introduction}
Eiger is a framework for the construction, analysis, and use of
statistical performance models of computer systems.  It is designed with
the goals of flexibility and understandability above that of model
construction speed.  As such, the framework uses as little expert
knowledge about the modeled system as possible, as well as generalized
notions of independent and dependent metrics. The models it generates
are concise, explicit functional representations of the performance
metric, allowing for insight into those metrics that most contribute to
performance. As well, a set of additional analysis tools are provided to
aid users in validating, improving, and understanding the models they
generate. This document will present a brief lay out of key components
within Eiger, common utilization methodology, and resources for more
information.

\subsection{Modeling Procedure}
As a statistical modeling tool, Eiger requires training data. This data
is what informs the performance models, as well as guides their
construction and accuracy.  Eiger will generate performance models that
only follow the training data it is provided with; this implies that
Eiger must be trained on data representative of the types of models the
user wishes to generate. As a rule of thumb, the more data that Eiger
has to work with, the better quality the models. The pathway that Eiger
sets up for modeling has several phases:

	\begin{description}
		\item[Collection of Instrumented Data]
		Instrumentation API implementations in Python and C++ for sending collected input data to the Eiger internal data store.
		\item[Storage]
		MySQL database schema and integration with instrumentation
		APIs.
		\item[Generation]
		Flexible, semi-automated statistical model generation.
		\item[Analysis]
		Performance and structure visualization tools, as well as
		validation procedures.
		\item[Export]
		Standardized model object export, integration with SST/macro
		large-scale simulator.
	\end{description}

\subsection{Further Information}
For more information on the motivation, implementation details, 
experimental results, and exposition on future work, see the 
publication ``Eiger: A Framework for the Automated Synthesis of 
Statistical Performance Models" in the 1st Workshop on Performance 
Engineering and Applications, held in conjunction with the 2012 High 
Performance Computing (HiPC) conference. It is also included in 
this directory as \texttt{eiger-WPEA2012.pdf}.\\ \\
Should any questions or concerns arise, feel free to contact the
authors:
	\begin{quote}
	\centering
	Eric Anger	-	eanger@gatech.edu\\
	Andrew Kerr	-	akerr@gatech.edu
	\end{quote}
