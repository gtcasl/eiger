% author: Ben Allan
% inst: sandia national laboratories
% date: Oct 2012
\section{Non-SQL Application Programming Interface}
\label{sec:fakeapi}
This document defines the Non-SQL API to Eiger, for convenience named {\em fakeeiger}.

\subsection{Requirements}

\begin{itemize}
\item Provide as close to the SQL-based Eiger API as possible without using SQL where mysql is unavailable.

\item Permit multiple executions to be accumulated into the same data set.

\item Provide a mechanism to load the non-sql results into mysql.
 
\item Generate obvious errors (preferably at compile time) where incompatibilities exist.

\end{itemize}

\subsection{Differences with MySQL-based Eiger}

\subsubsection {Runtime behavior} 
Fakeeiger replaces interprocess communication (and possible database file reads)  with a lightweight in-process approximation of the database behavior and filesystem writes to a log file of the form "{\em fakeeiger.log.XXXXXX}", where the last 6 characters are constructed randomly. Fakeeiger.log is written as eiger object commits occur and contains the data needed to read a collected result set into a mysql database using the \texttt{eiger-loader} utility, which takes a list of files to parse as command line arguments.  

Because of the node-local, in-process nature of fakeeiger, it is not inherently thread-safe. Eiger calls should typically be made only from the lead thread when profiling OpenMp applications with fakeeiger. However, fakeeiger is capable of running in MPI parallel environments and files written concurrently should be parsed correctly. Useful analysis of compute-bound code sections can still occur under these conditions.

\subsubsection{Identifiers} Using MySQL permits nearly arbitrary combining of performance experiments into a single database. Part of this functionality involves providing unique integer identifiers for metadata items. In some cases, creating the same kind of item by the same name results in a database query to check if the metadata combination pre-exists; if the item exists, the integer ID is read from the database rather than generated. Obviously, this cannot be done efficiently without MySQL (or even with MySQL on very slow parallel file systems). The fakeeiger solution is to recognize that a restricted, but still useful, pattern of Eiger use is for a given application (across distinct runs) to declare all the metadata in the same order every time, resulting in the same integer identifiers. Within a run, where metadata definitions are repeated, a map is used to ensure appropriate common identifiers are used. Where identifiers must be distinct across all runs, offsets are stored at the end of a run and read at the beginning as previously noted. 

\subsubsection{Linking} As the Eiger API is based on concrete classes rather than a functional interface, fakeeiger provides an alternate library, {\em libfakeeiger}, which implements all the data writing functions of libeiger. Just link your program with fakeeiger and all Eiger API calls will be intercepted appropriately.

\subsubsection{Output} The output always goes to fakeeiger.log.XXXXXX, where the last six characters are random, in the directory from which the eigerized program is executed. The output may be edited at the top to adjust the name of the database or other parameters before loading into the database if needed. Loading from fakeeiger.log is done by running \texttt{eiger-loader}.



\subsection{Possible improvements}

\begin{itemize}
\item[Format] Fakeeiger uses a human-readable, line-oriented format. A speed improvement (at the expense of debugging ease) would be to convert to a binary file format in fakeeiger.log.
\item[Thread safety] The fakeeiger api is not intentionally thread-safe. To date, it has only been used to collect data from single processes or from multiple threads/processes where a leader handles performance data logging. A simple improvement would be to open one log-file per process, perhaps by suffixing the process id to fakeeiger.log. To support multithread or multi-process use, a revised FakeEigerLoader class will be needed to coordinate merging of various IDs. Alternately, analysis could be done rankwise and fakeeiger would need to be slightly revised to incorporate process rank into the log filename.
\end{itemize}
