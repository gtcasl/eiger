\section{Implementation Details}
\label{sec:imp}
This document walks through the implementation details of the Eiger components,
giving an overview of major structures and programming practices.

\subsection{Class Hierarchy}
Eiger employs a rigid structure for the description of training data. This
structure, enforced by the API specification as well as the interal storage
mechanism, is intended to be as flexible and understandable as possible, so
that users are able to formulate their model generation in whichever way is
easiest. This document will lay out the basic hierarchy and interactions
between classes; more information can be found in the API specification.

\begin{description}
\item[DataCollection]Performance models are built from data contained in a
DataCollection. A DataCollection serves as an identifier for a set of training
data points, known as {\em Trials}.

\item[Trial] A Trial represents a single run of a program to collect training data.
Each Trial is a single run of a particular {\em Application}, fed with data from a
particular {\em Dataset}, executed on a particular {\em Machine}.
A Trial serves as a unifying structure under which all
Metrics are contained, including those from the Machine, the Application, and
the performance metric.

\item[Execution] An execution represents the individual instrumentation tools used
to collect metrics from a single Trial. The union of Metrics from all of a
Trial's Executions comprize the entire set of Metrics that describe that Trial.
This allows for running several different tools (emulator, simulator, etc) to
collect metrics that will all go together to describe a Trial.

\item[Application] An Application is a program run to collect training data. It can
be a full application, a kernel, or a microbenchmark.  Several different input
Datasets can be run through the same Application, i.e. different sized input
matrix Datasets can be run through the same matrix multiplication Application.

\item[Dataset] A Dataset describes a particular set of input data run through an
Application. A Dataset may not be shared across several Applications.
DeterministicMetrics are associated with a Dataset; any run of the same
Application with the same Dataset should result in the same values for
DeterministicMetrics.

\item[Machine] The Machine class specifies the hardware upon which an Application is
run. This class is used to contain all Metrics that are invariant across
Applications that run on the same hardware.

\item[Metric] A Metric describes a measurable quantity of the system. 
There are several types of Metrics:
	\begin{description}
	\item[result] Performance metric the model will try to predict.  Runtime,
		energy use, etc.
	\item[deterministic] Any value that is invariant across executions of a
		particular execution. Static instruction counts, input 
		data size, etc.
	\item[nondeterministic] Values that may change from execution to execution.
		This includes measures such as dynamic instruction counts, cache miss
		rates, etc.
	\item[machine] Values that pertain only to the Machine an Application is run
		on. Memory capacity, clock speeds, etc.
	\item[other] Miscellanea. 
	\end{description}

\item[DeterministicMetric] This class acts as a wrapper for a single result of a
measured Metric that has deterministic type. Each Dataset will have one
DeterministicMetric for every Metric of deterministic type.

\item[NondeterministicMetric] Analogous to DeterministicMetrics but for Metrics
of nondeterministic type. This includes the result type, as by definition
any performance result is dependent upon the execution of the workload on
hardware.

\item[MachineMetric] Analogous to DeterministicMetric but for Metrics of machine
type. 
\end{description}

\subsection{Internal Data Store}
By design the API abstracts away the requirement for any particular
database technology, although at this time this interface is only
implemented for the MySQL database. There is a table for each of the major
classes, as well as extra tables used for model serialization. Eiger does
not require that the database be stored locally; rather, the API provides a
way to indicate the IP address or hostname of the machine, allowing for
nonlocal data storage.  Further details can be found in the API
documentation in Section \ref{sec:api}. The schema describing the database is in
\texttt{eiger/database/schema.sql}. 

\subsection{Model File Format}
For integration with the SST/macro simulator, Eiger can serialize its
performance models to files. Each file is formatted so that it can be read
by the appropriate module within SST/macro, however there is nothing
inherent to this format that prevents it from being used by different
sources. More details on the exact file format can be found in
\texttt{eiger/PerformanceModel.py} under the \texttt{toFile()} function. 
See the FAQ in Section \ref{sec:faq}
for details on exporting performance models.
